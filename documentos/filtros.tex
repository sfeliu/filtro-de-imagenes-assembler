\subsection{Blit}

Esta operaci\'on recibe m\'as par\'ametros que las dem\'as; adem\'as de tener las im\'agenes fuente y destino (de tama\~no w x h), 
se recibe una imagen adicional \textbf{blit} tambi\'en con su respectivo tama\~no (bw x bh), que se utilizar\'a a modo de m\'ascara.
Se cumple que $bw \leq w \wedge bh \leq h$.\\

La aplicaci\'on de este filtro consiste en generar una nueva imagen combinando la original con el \textbf{blit}.
Un color de la imagen \textbf{blit} es considerado como transparente,
para que en la combinaci\'on algunos pixeles del \textbf{blit} queden por encima de la imagen original y otros no se vean.

Para este trabajo pr\'actico, la imagen \textbf{blit} ser\'a una imagen de Per\'on.
Decimos que una im\'agen ha sido ``Peronizada'' cuando en su extremo superior derecho podemos ver a Per\'on agitando su brazo.
% La aplicaci\'on de este filtro consiste en generar una nueva imagen combinando la original con la de Per\'on (blit) para as\'i 
% obtener una versi\'on ``Peronizada'' de la im\'agen fuente.

\vskip 8pt
Para cada p\'ixel $p$ en la imagen de Per\'on (blit):\\
$dst(p) = \begin{cases}
    src(p) & \mathrm{si \;} blit(p) \mathrm{\; es \; de \; color \; magenta, \; es \; decir \; sus \; colores \; son \;} (255, 0, 255)\\
    blit(p) & \mathrm{si \; no}
\end{cases}$ \\

Es decir que si el p\'ixel en la imagen de Per\'on es magenta, entonces el nuevo p\'ixel tendr\'a los colores del p\'ixel correspondiente 
en la imagen original; y si no los colores del p\'ixel en la imagen de Per\'on. \\

Las columnas y filas que no se puedan procesar de esta manera, es decir, 
las primeras $h - bh$ filas; y de las filas restantes, las primeras $w - bw$ columnas; 
quedar\'an igual que como estaban originalmente. \\

Nota: como debe valer la restricci\'on $bw \leq w \wedge bh \leq h$, esta operaci\'on 
no podr\'a ser aplicada a im\'agenes cuyo tama\~no sea menor a 89 p\'ixeles de ancho 
x 128 de alto.

%M\'as informaci\'on: http://en.wikipedia.org/wiki/Bit_blit

\subsubsection*{Implementación y uso}
\noindent Funciones: \code{blit_c}, \code{blit_asm}

\noindent Parámetros:
\begin{enumerate}[-]
\item \code{unsigned char *blit}: Imagen que se va a blitear
\item \code{int bh}: Alto de la imagen que se va a blitear
\item \code{int bw}: Ancho de la imagen que se va a blitear
\item \code{int b_row_size}: Bytes por fila de la imagen que se va a blitear
\end{enumerate}


% -----------------------------------------------------------------
% -----------------------------------------------------------------

\subsection{Monocromatizar norma $\inf$ } %(normas $1$ e $\inf$)}

Para convertir una imagen a escala de grises debemos contar con una funci'on 
que sea cap'az de monocromatizar una imagen a color. La funci'on m'as sencilla 
para hacer esto es:

\begin{center}
  $f(p) = \sqrt[\epsilon]{\alpha R^\epsilon + \beta G^\epsilon + 
    \gamma B^\epsilon}$
\end{center}

La funci'on $f$ se aplica a cada p'ixel $p$ de la imagen; $R$, $G$, $B$ son 
sus componentes de color, $\alpha$, $\beta$ y $\gamma$ son coeficientes entre 
$0$ y $1$, y $\epsilon$ es un exponente entre $1$ e $\infty$.\\

En nuestro caso se implementar'a la funci'on de conversi'on a escala de grises 
para un casos extremo, donde $\epsilon$ en la funci'on vale  
$\epsilon = \infty$. Adem'as, se fijar'an los par'ametros restantes en 
$\alpha = 1/4$, $\beta = 1/2$ y $\gamma = 1/4$. Luego, tenemos la siguiente
funcion para monocromatizar:

%para los casos extremos, donde $\epsilon$ en la funci'on vale $\epsilon = 1$ o 
%$\epsilon = \infty$. Adem'as, se fijar'an los par'ametros restantes en 
%$\alpha = 1/4$, $\beta = 1/2$ y $\gamma = 1/4$. Luego, tenemos dos posibles
%funciones para monocromatizar:

%\begin{center}
%  $I_{out\_uno}(p) = \frac{(R + 2G + B)}{4} (\epsilon = 1)$
%\end{center}

\begin{center}
  $I_{out\_infinito}(p) = max(R, G, B) (\epsilon = \infty)$
\end{center}


\subsubsection*{Implementación y uso}
\noindent Funciones: \\
\code{monocromatizar_inf_c}, \code{monocromatizar_inf_asm},
 
\noindent Parámetros: ninguno



% -----------------------------------------------------------------
% -----------------------------------------------------------------

\subsection{Efecto Ondas}

Deber'an implementar en \textbf{SSE} la funci'on ``ondas''. Esta combina la 
imagen original con una imagen de ondas, dando tonos m'as oscuros y m'as claros 
en forma de onda. Estas se generan desde el centro de la imagen hacia 
sus bordes de manera conc'entrica.\\

Para simplificar la tarea de programaci'on se proporciona una implementaci'on 
en C de este efecto en el archivo ``ondas\_c.c''.\\

El procedimiento a realizar es el siguiente:

\begin{algorithm}[H]
  \begin{algorithmic}[1]
    \FORALL{pixel ubicado en la posici'on $\mathbf{(x, y)}$}
      \STATE $d_x \gets x - x_0$

      \STATE

      \STATE $d_y \gets y - y_0$

      \STATE

      \STATE $d_{xy} \gets \sqrt{d_{x}^2+d_{y}^2}$

      \STATE

      \STATE $r \gets \frac{(d_{xy} - RADIUS)}{WAVELENGTH}$

      \STATE

      \STATE $a \gets \frac{1}{1 + (\frac{r}{TRAINWIDTH})^2 }$

      \STATE

      \STATE $t \gets ( r-floor(r) ) \cdot 2 \cdot \pi - \pi$

      \STATE

      \STATE $prof \gets a \cdot (t - \frac{t^3}{6}+\frac{t^5}{120}-\frac{t^7}{5040})$

      \STATE

      \STATE $pixel = prof \cdot 64 + I_{src}(x, y)$    

      \STATE

      \STATE $I_{dst}(x, y) = saturar(pixel)$
    \ENDFOR
  \end{algorithmic}
  \caption{$ondas (I_{src}, I_{dst}, x_0, y_0)$}
  \label{alg:ondas}
\end{algorithm}

donde:

\begin{itemize}
  \item $x_0$ e $y_0$ representan la posici'on donde est'a centrada la onda,
  \item $RADIUS$, $WAVELENGTH$ y $TRAINWIDTH$ son constantes que definen la 
  forma de la onda y
  \item $saturar(x)$ es una funci'on que retorna $0$ si $x$ es menor $0$, $255$
  si es mayor a $255$ y $x$ en cualquier otro caso.
\end{itemize}

\subsubsection*{Implementación y uso}

\noindent Funciones: \code{ondas_c}, \code{ondas_asm},

\noindent Parámetros: ninguno

\noindent Ejemplo de uso: \code{ondas -i c lena.bmp}




% -----------------------------------------------------------------
% -----------------------------------------------------------------

\subsection{Edge}

Puede definirse como un borde a los p\'ixeles donde la intensidad de la imagen cambia de forma abrupta. 
Si se considera una funci\'on de intensidad de la imagen, entonces lo que se busca son saltos en dicha funci\'on.
La idea b\'asica detr\'as de cualquier detector de bordes es el c\'alculo de un operador local de derivaci\'on. \\
 

Vamos a usar en este caso el operador de Laplace, cuya matriz es

$$ M = \left(
\begin{matrix}
    0.5 & 1 & 0.5 \\
    1 & -6 & 1 \\
    0.5 & 1 & 0.5
\end{matrix}
\right)$$


Para obtener los bordes, se posiciona el centro de la matr\'iz de 
Laplace en cada posici\'on $(x, y)$ y se realiza la siguiente operaci\'on

$$dst(x, y) = \sum_{k = 0}^2 \sum_{l = 0}^2 src(x + k - 1, y + l - 1) * M(k, l)$$

Es decir $dst(x, y) = $
\begin{center}
    $\begin{matrix}
        src(x - 1, y - 1) * M_{00} & + & src(x - 1, y) * M_{01} & + & src(x - 1, y + 1) * M_{02} & +\\
        src(x, y - 1) * M_{10} & + & src(x, y1) * M_{11} & + & src(x, y + 1) * M_{12} & +\\ 
        src(x + 1, y - 1) * M_{20} & + & src(x + 1, y) * M_{21} & + & src(x + 1, y + 1) * M_{22} &
    \end{matrix}$
\end{center}

Este filtro opera sobre imágenes en escala de grises (1 componente de color por píxel).
En los bordes que no se pueden procesar según la fórmula anterior (por estar a los costados de la
imagen), aplica la fórmula $dst(x, y) = src(x,y)$.

\subsubsection*{Implementación y uso}

\noindent Funciones: \code{edge_c}, \code{edge_asm}

\noindent Parámetros: ninguno




% -----------------------------------------------------------------
% -----------------------------------------------------------------

\subsection{Temperatura}

    El filtro temperatura toma una imagen fuente y genera un
    efecto que simula un mapa de calor. El filtro toma
    los tres componentes del cada pixel, los suma y divide por 3, y califica a eso
    como la temperatura $t$.

    \begin{center}
    $\mathsf{t}_{(i,j)} = \lfloor(\mathsf{src}.r_{(i,j)} + \mathsf{src}.g_{(i,j)} + \mathsf{src}.b_{(i,j)}) / 3\rfloor$
	\end{center}
    
    En función de la temperatura, se determina el color en la imagen destino. 
    Para evitar diferencias con los resultados la cátedra, la 
    temperatura debe truncarse y guardarse en una variable de tipo entero.

    \begin{center}
	\begin{displaymath}
	\mathsf{dst}_{(i,j)}<r,g,b> = \left\{
	\begin{array}{l l}
				<0,0, 128 + t \cdot 4> & \text{si }t < 32\\
				<0, (t - 32) \cdot 4, 255> & \text{si }32 \le t < 96\\
				<(t-96) \cdot 4, 255, 255 - (t-96) \cdot 4> & \text{si }96 \le t < 160\\
				<255, 255 - (t - 160) \cdot 4, 0> & \text{si }160 \le t < 224\\
				<255 - (t - 224) \cdot 4, 0 , 0> & \text{si no} \\
	\end{array}
	\right.
	\end{displaymath}
	\end{center}
	
    \newpage




\subsubsection*{Implementación y uso}

\noindent Funciones: \code{temperature_c}, \code{temperature_asm},

\noindent Parámetros: ninguno

\noindent Ejemplo de uso: \code{temperature -i c lena.bmp}


% -----------------------------------------------------------------
% -----------------------------------------------------------------
